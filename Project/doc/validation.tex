\section{Validierung}
Folgende Auflistung beschreibt die implementierten Validierungen:
\begin{itemize}
	\item\textbf{\emph{Leere Beschreibung}}: Es wurde eine Validierung eingeführt, die auf eine leere Beschreibung eines Moduls prüft, damit ein \emph{Quickfix} eine initiale Beschreibung erstellen kann.
	
	\item\textbf{\emph{Duplikate Namen}}: Da die Namen vom Datentyp \emph{ID} sind, wurde zusätzlich eine Prüfung eingeführt, ob die Namen bereits vergeben wurden.
	
	\item\textbf{\emph{Camel case Namen}}: Da die Namen vom Datentyp \emph{ID} sind, wurde zusätzlich eine Prüfung eingeführt, ob die Namen in \emph{camel case} sind.
	
	\item\textbf{\emph{Duplikate Locale Einträge}}: Es wurde eine Validierung eingeführt, die prüft ob es Duplikate der \emph{Locale} für ein \emph{Localized} gibt. 
	
	\item\textbf{\emph{Duplikate Referenzen}}: Es wurde eine Validierung eingeführt, die prüft ob es Duplikate bei den gesetzten Referenzen gibt.
\end{itemize}
\ \newline
Mit einer Grammatik kann nicht beschrieben werden, dass es keine Duplikate bei den verwendeten Namen von Objekten in einer Auflistung gibt. Das ist so, da es sich hierum Semantik handelt und nicht Grammatik, ob ein Name nur einmalig oder mehrmalig vergeben werden darf.
\newline
\newline
Wenn das Attribut Name bereits als \emph{ID} festgesetzt wurde, dann kann nicht zusätzlich eine Terminalregel angewendet werden, die sicherstellt, das der Name auch z.B.: in \emph{camel case} ist. 
\newline
\newline
Wenn eine Auflistung definiert wurde wie \emph{(localizedEnums+=[Localized]+)}, dann kann mit der Grammatik nicht verhindert werden, das Duplikate bei den Referenzen angeben werden, da es sich hier auch um Semantik handelt. Das Attribut \emph{localizedEnums} wird als Liste abgebildet.
\newline
\newline
Das ein \emph{Localized} für eine \emph{Locale} nur einmal einen Eintrag definieren kann, ist eine Semantik, die auch nicht über die Grammatik abgebildet werden kann.